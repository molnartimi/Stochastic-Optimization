\chapter{Összefoglalás}

Az optimalizálásra az élet rengeteg területén van igényünk. Láthattuk azonban, hogy ennek a módjának a kiválasztásánál nagyon körültekintően kell eljárnunk, hiszen más és más problémakörökre lettek kitalálva az egyes algoritmusok. 

A sztochasztikus rendszerek esete egy különösen nehéz terület, nem nagy meglepetés tehát, hogy az algoritmusok nem arattak rögtön elsöprő sikert. Az első lépés azonban ezzel készen van. Láttuk, hogy a Bayesi optimalizálás nyújtja azokat a lehetőségeket, melyeket kihasználva hatékonyan áthidalhatjuk a problémáink fő nehézségeit: a függvényhívások minimalizálását, és az ismeretlen, nem értelmezhető területek figyelembe vételét. Az idő ezek függvényében már csak egy másodlagos szempont -- inkább legyen egy egy óra alatt lefutó pontos megoldásunk, mint egy néhány perc alatt kiszámolt használhatatlan eredményünk. Ezeknél az algoritmusoknál ugyanis nem a sok függvényhívás, hanem a Gauss folyamatok kezelésének a bonyolultsága foglalja magába a futási idő oroszlánrészét.

A továbbiakban az ígéretesnek tűnő út a klasszifikáció finomítása. Bár a modelljeinkre a büntetőérték alkalmazásával a másik két Bayesi implementáció jobban teljesített a kiszámítható és nem értelmezhető pontok arányában, amennyiben nagyobb modellekkel szeretnénk dolgozni, ez nem lesz járható út. Nem tudhatjuk ugyanis biztosra, hogy tudunk-e olyan nagy büntető értéket adni, mely jóval nagyobb a többi kiszámított pont függvényértékénél.

Ezen kívül nagy vágyunk az algoritmusok kiterjesztése a hiperparaméterek optimalizálására is. Mivel ez is egy nagy fajsúlyú probléma az algoritmusok alkalmazásánál, nagyon jó lenne, ha ezt a súlyt levehetnénk a fejlesztők válláról.

A feladat megoldása tehát nem lehetetlen. Sok munkát, sok mérést, sok kutatómunkát és sok próbálkozást igényel, de a végeredmény egy remélhetőleg sok helyen sikerrel alkalmazható, nagy hasznosságú alkalmazás lesz. Ezt pedig meg is fogjuk próbálni elérni.