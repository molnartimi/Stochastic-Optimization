% !TeX spellcheck = hu_HU
% !TeX encoding = UTF-8
% !TeX program = xelatex
%----------------------------------------------------------------------------
\chapter{Implementációk}
\label{sec:implementaciok}
%----------------------------------------------------------------------------
Az alábbiakban \aref{sec:algoritmusok}. fejezetben kifejtett algoritmusok implementációit részletezem. A nemkorlátos optimalizáló algoritmusok Java, a Bayesi optimalizálók viszont Python nyelven készültek. Ennek csupán annyi a magyarázata, hogy a kutatás kezdetén még nem vált világossá, mely algoritmusokhoz milyen nyelveken áll rendelkezésre a legtöbb létező implementáció, amiket felhasználva a lehető legjobb eredmények érhetőek el.

\subsubsection{Stochastic PetriDotNet megoldó keretrendszer integrálása}
A program lelkét a tanszéken fejlesztett SPDN\footnote{\url{https://inf.mit.bme.hu/research/tools/petridotnet}} % TODO Appendixben róla pár szó?
keretrendszer adja. A futtatásával, és a vele való folytonos kommunikálással kapjuk meg az egyes paraméterlekötések esetén a reward függvények értékét, melyből az összesített négyzetes hibát adó célfüggvényünket számítjuk. Lehetőségünk van a reward függvények egyes paraméterek szerint vett parciális deriváltjának a kiszámítására is.

Mint már említettük, a megoldó bizonyos esetekben nem képes kiszámítani egy adott paraméter lekötéshez tartozó reward függvényértékeket. Ekkor a futtatott alkalmazás kimenetén egy hibát olvashatunk, mely arról tájékoztat, hogy a megoldó algoritmus az iterációi során NaN megoldásba futott. Az egyes algoritmusok feladata, hogy ezt az esetet -- lehetőségeikhez mérten -- kezeljék.

\subsubsection{Modellek itegrációja}
Minden optimalizálást végző osztályom az optimalizálandó modell beállításával kezdődik. A programkódban a modellekhez az alábbi információkat tároljuk:
\begin{itemize}
	\item \textbf{Modell} négy betűből álló \textbf{azonosítója}.
	\item Modellt tartalmazó (.pnml kiterjesztésű) \textbf{fájl neve}.
	\item \textbf{Paraméterek} nevei.
	\item Paraméterekhez tartozó \textbf{default értékek}. Ezzel futtatva a megoldó keretrendszert kapjuk meg azon értékeket, melyeket követelményként fogalmazunk meg a reward függvények értékeinek. Tehát ez az a pont, melyet az optimalizálás során szeretnénk minél jobban megközelíteni. Valós esetekben ezt nem ismerjük, most csak a mérések eredményének ellenőrzéséhez használjuk.
	\item Paraméterekhez tartozó alsó és felső \textbf{korlátok}. A Nem korlátos algoritmusok esetén ezt csak ahhoz használjuk, hogy milyen tartományban válasszuk a véletlenszerű kezdőpontokat, a futás során azonban a tartományon kívülre is eljuthat az algoritmus. A Bayesi algoritmusok a határok között maradnak.
	\item \textbf{Reward függvények} nevei.
	\item Reward függvényekhez tartozó \textbf{elvárt értékek}.
\end{itemize}

Az adott modellel az objektum létrehozza a saját SPDN példányát, mely az optimalizálandó célfüggvényt szolgáltatja.

%----------------------------------------------------------------------------
\section{Nemkorlátos optimalizáló algoritmusok}
%----------------------------------------------------------------------------

\subsubsection{Modellek - Model.java}
A Java kódok esetében a modellek adatai a Model enumban találhatóak. Getter függvényeivel kérdezhetőek le az információk, getRandomPoint() metódusával a paraméterek értelmezési tartományából egy véletlen pont generálható, a getRandomVelocity() függvénnyel pedig a részecske raj optimalizációknál használt véletlen sebességvektor.

\subsubsection{SPDN - SPDN.java, SPDNResult.java}
A megoldó keretrendszerrel való kommunikációt a tanszéken írt Java spdn-interactive segédalkalmazás szolgáltatja. % TODO hogy hivatkozzak Kristófra?
Ezt, és a többi függőséget is a Maven kezeli. Az én SPDN osztályom a megfelelő objektumok hívásával állítja elő a kívánt eredményt. Implementálja a DiffFunction interfészt, melyre az L-BFGS algoritmus miatt lesz szükség. A megoldó hibája esetén SpdnException objektum dobódik. % Ezt úgy kezeljük, hogy minden sikeres számítás esetén elmentjük a kiszámított függvényértéket. Hiba esetén ennek az értéknek az x-szeresét adjuk vissza, ezzel kikerülve azt, hogy minden modell esetén egy konstans hibaértéket adjunk vissza, mely a szükségesnél jobban kilógna a többi számított érték közül.
% TODO hibakezelés?
% Ötlet: SPDN-ben tárolni az előző jó eredményt, hiba esetén ennek x-szeresét visszaadni\\\\
Metódusai:
\paragraph{double f(double[] variables):} Adott paraméterek esetén visszadja a célfüggvény értékét. Célfüggvényünk egy kissé eltér \aref({eq:celfgv}) képlettől. Nem korlátos algoritmusaink futása során olyan pontok kiszámítására is lehet igényünk, ami a modellünk esetében nem értelmezett. Gyakran előforduló eset, hogy az egyes paraméterek negatív értékével szeretnénk kiszámoltatni a megoldóval a függvényértéket, ami sok modellparaméter esetében értelmetlen. Ennek kiküszöböléséhez, hogy mégis megadjuk a lehetőséget az algoritmusnak a tovább számoláshoz, a paramétereket mint hatványkitevőket használjuk fel. A megoldónak $e^p$ értékeket adunk a p paraméterekhez, a célfüggvényünk pedig így fog kinézni:
$$ f(p_1,p_2,...)=\sum_{i=1}^{rewards}\left(\hat{R}_i-R_i(e^{p_1},e^{p_2},...)\right) ^2$$
\paragraph[double f(RealVector variables]{double f(RealVector\footnote{Apache Common Math 3 API} variables):} Adott paraméterek esetén visszadja a célfüggvény értékét.
\paragraph{RealVector df(double[] variables):} Adott paraméterek esetén visszaadja a célfüggvény gradiensét.
\paragraph{RealVector df(RealVector variables):} Adott paraméterek esetén visszaadja a célfüggvény gradiensét.
\paragraph{ValueAndGradient calculate(double[] variables):} Egy ValueAndGradient objektumként adja vissza a függvényértéket és a gradienst is, a DiffFunction interfész metódusa.
\paragraph{int getDimension():} Az objektum által kezelt modell paramétereinek számát adja vissza.
\paragraph{RealVector convertPoint(RealVector v):} Az algoritmusok által adott végeredmény visszakonvertálását végzi, hogy ne az $e^p$ alakban írjuk ki a kapott optimális pontot.\\

Az algoritmusok által adott végeredményt az SPDNResult osztály kezeli, mely tartalmazza az algoritmus és az optimalizált modell négy betűből álló azonosítóját, az optimális pontot és az ottani függvényértéket, valamint a modell paramétereinek megnevezéseit. \\

Az algoritmus osztályok optimize() metódusát meghívva tudjuk azt a modellt optimalizálni, melyet az objektum létrehozásakor paraméterül átadunk. Visszatérési értéke egy SPDNResult objektum. A metódus paraméterei jelentősen befolyásolhatják az algoritmus hatékonyságát és eredményét. A könnyebbség kedvéért, ha valamely paraméterről nincsen információnk, típustól függően nullát vagy üres tömböt paraméterül adva, az algoritmus az előre beállított default paraméter értékkel fog lefutni.

\subsection{L-BFGS - LBFGS.java}
A Breeze könyvtár\footnote{\url{https://github.com/scalanlp/breeze}} számos algoritmus implementációját tartalmazza. Osztályai hatékony, gyors és generikus numerikus számításokat szolgáltatnak. Innen használtuk fel az L-BFGS algoritmust, melyhez, mivel Scala nyelven van írva, egy tanszéki segédalkalmazást használok a nyelvi konverzióhoz. Optimize metódusának paramétereit \aref{table:lbfgs}. táblázat mutatja.

\begin{table}
	\center
	\begin{tabular}{|ll>{\tabsorvege{\raggedright}\mbox{}}p{70mm}r|}
		\hline
		\textbf{Típus} & \textbf{Paraméter} & \textbf{Leírás} & \textbf{Default érték}\\
		\hline \hline
		int & \textbf{m} & Hesse mátrix becslésének pontosságát befolyásolja & 4\\
		\hline
		int & \textbf{maxIter} & Algoritmus futása során felhasznált egydimenziós optimalizálás maximális iterációszáma & 20\\
		\hline
		double & \textbf{tolerance} & Elfogadható maximális tolerancia & 0.001\\
		\hline
		double[] & \textbf{initPoint} & Algoritmus kezdőpontja & véletlen pont\footnote{Model.getRandomPoint() metódustól}\\
		\hline
		int & \textbf{restart} & Algoritmus újraindításának a száma, különböző véletlen kezdőpontokkal, a pontosabb megoldás megtalálásának érdekében & -\\
		\hline
	\end{tabular}
	\caption{LBFGS.optimize() metódus beállítható paraméterei}
	\label{table:lbfgs}
\end{table}


\subsection{Gradiens módszer - GradientDescent.java}
Optimize metódusának paramétereit \aref{table:gd}. táblázat mutatja.

\begin{table}
	\center
	\begin{tabular}{|ll>{\tabsorvege{\raggedright}\mbox{}}p{70mm}r|}
		\hline
		\textbf{Típus} & \textbf{Paraméter} & \textbf{Leírás} & \textbf{Default érték}\\
		\hline \hline
		double & \textbf{gamma} & Lépéshossz kezdeti értéke & 1\\
		\hline
		double & \textbf{tolerance} & Az algoritmus leáll, ha a pontban a gradiens hossza kisebb, mint a tolerancia & 0.001\\
		\hline
		double[] & \textbf{initPoint} & Kezdőpont & véletlen pont\footnote{Model.getRandomPoint() metódustól}\\
		\hline
		int & \textbf{restart} & Algoritmus újraindításának a száma & -\\
		\hline
	\end{tabular}
	\caption{GradientDescent.optimize() metódus beállítható paraméterei}
	\label{table:gd}
\end{table}

\subsection{Részecske raj optimalizációk}

\subsubsection{Homogén részecskék - ParticleSwarm.java}

Optimize metódusának paramétereit \aref{table:pso}. táblázat mutatja.

\begin{table}
	\center
	\begin{tabular}{|ll>{\tabsorvege{\raggedright}\mbox{}}p{70mm}r|}
		\hline
		\textbf{Típus} & \textbf{Paraméter} & \textbf{Leírás} & \textbf{Default érték}\\
		\hline \hline
		int & \textbf{swarmSize} & Részecskék száma & 20\\
		\hline
		int & \textbf{maxIter} & Iterációk száma & 20\\
		\hline
		double & \textbf{omega} & Új sebesség számításakor a régi sebesség súlya & 0.2\\
		\hline
		double & \textbf{fiParticle} & Új sebesség számításakor a részecske által ismert minimumpont súlya & 0.4\\
		\hline
		double & \textbf{fiGlobal} & Új sebesség számításakor a globálisan ismert minimumpont súlya & 0.8\\
		\hline
	\end{tabular}
	\caption{ParticleSwarm.optimize() metódus beállítható paraméterei}
	\label{table:pso}
\end{table}

\subsubsection{Gradens módszerrel ötvözve - ParticleSwarmWithGradientDescent.java}

A ParticleSwarm osztály leszármazottja, saját optimize metódusának paraméterei részben megegyeznek a részecske raj és a gradiens módszer paramétereivel. Összefoglalva \aref{table:psogd}.~táblázatban láthatóak.

\begin{table}
	\center
	\begin{tabular}{|ll>{\tabsorvege{\raggedright}\mbox{}}p{60mm}r|}
		\hline
		\textbf{Típus} & \textbf{Paraméter} & \textbf{Leírás} & \textbf{Default érték}\\
		\hline \hline
		int & \textbf{swarmSize} & Részecskék száma & 20\\
		\hline
		int & \textbf{maxIter} & Iterációk száma & 20\\
		\hline
		int & \textbf{gradientMaxIter} & Gradiens módszer lépésszáma iterációnként, az aktuális globális minimum pontból indulva & 5\\
		\hline
		double & \textbf{gamma} & Lépéshossz kezdeti értéke & 1\\
		\hline
		double & \textbf{omega} & Új sebesség számításakor a régi sebesség súlya & 0.2\\
		\hline
		double & \textbf{fiParticle} & Új sebesség számításakor a részecske által ismert minimumpont súlya & 0.4\\
		\hline
		double & \textbf{fiGlobal} & Új sebesség számításakor a globálisan ismert minimumpont súlya & 0.8\\
		\hline
	\end{tabular}
	\caption{ParticleSwarmWithGradientDescent.optimize() metódus beállítható paraméterei}
	\label{table:psogd}
\end{table}

\subsubsection{Méh algoritmus - BeesAlgorithm.java}

Optimize metódusának paramétereit \aref{table:bees}. táblázat mutatja.

\begin{table}
	\center
	\begin{tabular}{|ll>{\tabsorvege{\raggedright}\mbox{}}p{55mm}r|}
		\hline
		\textbf{Típus} & \textbf{Paraméter} & \textbf{Leírás} & \textbf{Default érték}\\
		\hline \hline
		int & \textbf{maxIter} & Iterációk száma & 20\\
		\hline
		double & \textbf{initRadius} & Kör kezdeti sugara, amin belül véletlen pontokat választunk a toborzáskor a vezető méhek körül & 0.5\\
		\hline
		double & \textbf{radiusSmallerRate} & 1-nél kisebb pozitív szám, iterációnként ezzel szorozva csökkentjük a keresési kör sugarát, ezzel egyre precízebb keresést biztosítva & 0.8\\
		\hline
		int & \textbf{scoutSize} & Méhek száma (toborzott méheket nem ide számítva) & 20\\
		\hline
		int & \textbf{bestBeesSize} & Legjobb méhek száma & 8\\
		\hline
		int & \textbf{eliteBeesSize} & Elit méhek száma & 3\\
		\hline
		int & \textbf{recruitedOfBestsSize} & Legjobb méheket követő, toborzott méhek száma & 5\\
		\hline
		int & \textbf{recruitedOfElitesSize} & Elit méheket követő, toborzott méhek száma & 10\\
		\hline
	\end{tabular}
	\caption{BeesAlgorithm.optimize() metódus beállítható paraméterei}
	\label{table:bees}
\end{table}

\subsection{Szimulált lehűtés - SimulatedAnnealing.java}

Optimize metódusának paramétereit \aref{table:sima}. táblázat mutatja.

\begin{table}
	\center
	\begin{tabular}{|ll>{\tabsorvege{\raggedright}\mbox{}}p{60mm}r|}
		\hline
		\textbf{Típus} & \textbf{Paraméter} & \textbf{Leírás} & \textbf{Default érték}\\
		\hline \hline
		double & \textbf{initTemp} & Rendszer kezdeti hőmérséklete & 100\\
		\hline
		double & \textbf{coolingRate} & 1-nél kisebb pozitív szám, ezzel szorozva csökkentjük az aktuális hőmérsékletet iterációnként & 0.9\\
		\hline
		double & \textbf{border} & A pont ekkora sugarú környezetében választunk új véletlen pontot & 0.3\\
		\hline
		double & \textbf{borderSmallerRate} & 1-nél kisebb pozitív szám, ezzel szorozva a csökkentjük a sugarat iterációnként, melyen belül ugrálunk & 0.95\\
		\hline
		int & \textbf{restart} & Algoritmus újraindításának a száma & -\\
		\hline
	\end{tabular}
	\caption{SimulatedAnnealing.optimize() metódus beállítható paraméterei}
	\label{table:sima}
\end{table}

%----------------------------------------------------------------------------
\section{Bayesi optimalizáció}

\subsubsection{Modellek}
A Python kódok esetében a modellek adatai egy Model nevű nevesített ennessel valósítottam meg, a Collections könytár Namedtuple osztályát használva.

\subsubsection{SPDN}
Az SPDN.exe megoldó programmal való kommunikációt hivatott elvégezni az SPDN osztály. A másik szálon futó alkalmazást a Python subprocess modul segítségével irányítjuk, a bemenetére adjuk a paraméterek értékeit, valamint a reward függvények beállításait, majd a kimenetéről beolvassuk azok eredményeit, amit aztán a célfüggvényünkké transzformálunk.

Lehetőségünk van a deriváltak kiszámítására is, ezt azonban a dolgozatom jelenlegi állapotában még használjuk.

NaN ERROR esetén egy SPDNException objektum dobódik, melynek kezelését az optimalizáló algoritmusok hivatottak kezelni, lehetőségeikhez mérten.
\\\\Hívható metódusai:
\paragraph{start(verbose=False):} A keretrendszer elindítását végzi. Opcionálisan a verbose flag igazra állításával a megoldóval való kommunikáció nyomon követhető a konzolon.
\paragraph{f(values):} Célfüggvény kiszámítása a listában kapott paraméter értékekkel.
\paragraph{df(values):} Célfüggvény gradiensének kiszámítása a listában kapott értékekkel.
\paragraph{close():} Megnyitott fájlok, futó folyamatok leállítása.

%----------------------------------------------------------------------------
\subsection{Scikit-learn csomaggal - MyBayesianOptimization}
A Scikit-learn\footnote{\url{http://scikit-learn.org/stable/}} egy egyszerű és hatékony eszközcsomag, melyet főleg adatbányászathoz és adatok analizálásához használhatunk. Nyílt forráskódú, és változatos területeken hasznosítható. Számos szolgáltatása közül a Gauss folyamatokkal való számolást használja ki a Bayesian-Optimization-with-Gaussian-Processes\footnote{\url{https://github.com/fmfn/BayesianOptimization}}, mely egy korlátos globális optimalizációs csomag Bayesi megoldással és Gauss folyamatokkal, Python nyelven. Célja nagy költségű függvények globális maximumának a megkeresése, ahol különösen fontos a kizsákmányolás és felderítés közötti egyensúly megtalálása.

\subsubsection{BayesianOptimization}

A BayesianOptimization a csomag fő osztálya, mely az optimalizálást végzi. Példányosításánál megadandó paraméterek:

\paragraph{f} (lambda **args: number) alakú függvény, melyet optimalizálunk
\paragraph{pbounds} \{'x': (x\_min, x\_max), ...\} alakú szótár
az f függvény bemenő paramétereinek nevével, és azoknak alsó és felső korlátaival
\paragraph{verbose} flag beállításával az optimalizálás lépései a konzolra logolódnak.\\\\
Hívható metódusai:

\paragraph{initialize(points\_dict):} arra ad lehetőséget, hogy az általunk ismert függvényértékeket még az optimalizálás előtt betöltsük a Gauss folyamat közelítő modelljébe. \{'target': [f1,f2,..], 'x':[x1,x2,..], ...\} alakban
\paragraph{explore(points\_dict):} megadhatunk pontokat a függvénytérben, melyeket rögtön az algoritmus futása elején kiszámol a program, ezzel segítve, hogy plusz információval segíthessük az optimalizálást, ha rendelkezünk ilyennel. \{'x':[x1,x2,..], ...\} alakban
\paragraph{maximize(init\_points, restarts, n\_iter, acq, **gp\_params):} Az optimalizálást végző metódus. Paraméterei a modell felállításához használt véletlenszerű kezdőpontok száma, hogy hány véletlen kezdőértékkel futtassa újra a nyereség függvény optimalizálását végző segédalgoritmust, hogy hány iterációt végezzen el, és hogy milyen nyereség függvényt alkalmazzon. Tetszőlegesen további paraméterek is megadhatók a Gauss folyamat még precízebb testreszabásához.\\

\Aref{subsec:bayes}. fejezetben említett mindhárom nyereségfüggvény implementálva van a segédobjektumai között. A maximumkereséshez véletlen mintavételezést és a SciPy\footnote{Pythonban írt nyílt forráskódú csomag matemtaikai, mérnöki és tudományos problémákhoz. \url{https://www.scipy.org}} csomag minimize() metódusának L-BFGS-B variánsát alkalmazza.

A nyereség függvények paraméterei az xi és az alsó biztos határ esetében a kappa. Ezekkel befolyásolhatjuk mi magunk a kizsákmányolás és felderítés megoszlását, aminek a hatását részletesebben \aref{sec:meresek}. fejezetben láthatjuk.

\subsubsection{MyBayesianOptimization}
Az általam implementált MyBayesianOptimization osztály példányosít egy SPDN objektumot, és ezt optimalizálhatjuk az optimize() metódussal. Megadható paraméterei megegyeznek a BayesianOptimization maximize() metódusának paramétereivel, a hívandó függvény kivételével, ugyanis azt az SPDN f() metódusának meghívásával kapjuk, ennek azonban a -1-szeresét kell átadnunk paraméterként, hiszen mi nem a függvényünk maximumát, hanem minimumát keressük.

A BayesianOptimization megvalósítása lehetővé teszi ugyan a paraméterek korlátainak megadását, egyéb korlátkezelésre azonban nincs módunk. Ezért azokban az esetekben, amikor a célfüggvényünk SPDNException hibát dob, egy büntető értéket jelölünk meg függvényértéknek az adott pontban. Ez a büntetőérték tetszőlegesen nagy lehet, ezzel próbálva informálni a Gauss folyamatot, hogy ez egy nem megfelelő terület, ne a közelébe keresse az optimális pontot.

\subsection{TensorFlow könyvtárral - MyGPflowOpt}
A TensorFlow\footnote{\url{https://www.tensorflow.org}} egy nyílt forráskódú könyvtár adatfolyamokkal való numerikus számításokhoz. Egy gráffal dolgozik, melynek csomópontjai a matematikai műveletek, élei pedig a többdimenziós adattömbök, a tenzorok. A gépi intelligenciával foglalkozó problémakörök megoldásának nagy tárházát kínálja.

A GPflowOpt\footnote{\url{https://github.com/GPflow/GPflowOpt}} egy Pythonban írt könyvtár Gauss folyamatokkal való Bayesi optimalizációhoz. Része a GPflow\footnote{\url{https://github.com/GPflow/GPflow/}} csomag, mely a Gauss folyamatokat hivatott kezelni. A TensorFlowt felhasználva teszi lehetővé a számítások felgyorsítását, skálázhatóságát, és nagy segítséget nyúlt a nyereség függvények kiszámításában is, mivel mentesíti a programot a gradiens számításának implementálási nehézségeitől.

\subsubsection{GPflowOpt}

A GPflowOpt fő osztálya az Optimizer ősosztás, melynek a gyereke a BayesianOptimizer. Példányosításánál megadandó paraméterek:
\paragraph{domain:} függvényparaméterek értelmezési tartományai
\paragraph{acquisition:} nyereség függvény objektum
\paragraph{optimizer:} opcionális paraméter, Optimizer osztály leszármazottja, a nyereség függvény optimalizálásához. Ha nem adjuk meg, a SciPy könyvtár optimize.minimize() metódusát használja.
\paragraph{initial:} Mely pontokat kiértékelve állítsa fel az algoritmus a Gauss folyamat kezdeti modelljét.
\paragraph{scaling:} Opcionális paraméter, default értéke igaz. Ha igaz, a bemenetek és kimenetek normalizáltak.
\paragraph{hyper\_draws:} Opcionális paraméter, lehetővé teszi hiperparaméterek specifikálását. Alapértelmezett működés esetén becsült értékeket használunk. A paramétert n értékre beállítva a valószínűségi becslés eloszlásából % TODO likelihood
szerzünk n paramétert.
\paragraph{callback:} Opcionálisan megadható egy metódus, mely meghívódik minden modellfrissítés után. Ezzel lehetőséget nyújt a fejlesztőknek további funkcionalitásokat, módosításokat beiktatni az algoritmus menetébe.\\\\
Metódusok:
\paragraph{optimize(objectivefx, n\_iter):} a kapott függvényt optimalizálja n\_iter iterációszámban, melynek default értéke 20.\\

A Domain osztály reprezentálja a függvények paramétereit. Belőle öröklődik a Parameter osztály, abból pedig a ContinuousParameter, amit nekünk használni kell. Példányosításánál megadandó paraméterek:
\paragraph{label} paraméter neve,
\paragraph{lb} alsó korlát,
\paragraph{ub} felső korlát,
\paragraph{xinit} opcionális kezdőérték, default értéke a tartomány közepe.\\

Amiről eddig még nem volt szó, az az optimalizáció kezdetén a függvénytér mintavételezésének a módszere. A GPflowOpt ebből is több variációt nyújt, a Design ősosztály leszármazottjai formájában: 
\begin{itemize}
	\item RandomDesign(size, domain): N véletlenszerűen kiválasztott pont a megadott tartományban.
	\item FactorialDesign(level, domain): k-szintű felosztás minden dimenzió esetén, ezzel egy nagy rácshálóra bontva a teret.
	\item LatinHyperCube(size, domain, max\_seed\_size): pontok közötti minimális távolságot optimalizálja, a teret adott méretű rácshálóra bontva, majd minden sorból és oszlopból csak egy mintát vételezve. Opcionális paraméter a tér felbontásánál alkalmazott maximális tartományszélesség.
\end{itemize}

A Gauss folyamat modelljét a GPflow csomag GPModel osztályának GPR leszármazottja állítja elő.\\\\
Paraméterei:
\paragraph{X} NxD méretű mátrix, ahol N a kezdőpontok száma, D a függvény dimenziója,\\
\paragraph{Y} Nx1 méretű mátrix, a kezdőpontok kiértékelt függvényértékeivel,\\
\paragraph{kern} Kernel objektum,\\
\paragraph{mean\_function} opcionálisan megadható várhatóérték függvény, alapértelmezett értéke a konstans 0.\\

A GPflowban számos kernel függvény implementálva van, melyek közös absztrakt őse a Kern osztály. Többek között megtalálhatóak \aref{subsec:bayes}. fejezetben említettek is:
\begin{itemize}
	\item RBF: négyzetes exponenciális kernel
	\item Exponential: exponenciális kernel
	\item Matern12
	\item Matern32
	\item Matern52
\end{itemize} 

Ezeknek a kombinálására is ad lehetőséget, összeadva vagy összeszorozva két kernel objektumot, a tulajdonságaik együtt jutnak érvényre a számításokban. Ezzel kellően nagy teret engedve a fejlesztőnek a neki megfelelő kernel függvény kialakításához. Paraméterük többek között a bemenetek dimenziója. Többdimenziós esetben lehetőségünk van az ARD\footnote{Automatic Relevance Determination} opcióval megadni, hogy a dimenziók egymástól függetlenek.

A GPflowOpt a nyereség függvényekből is biztosítja a legnépszerűbbeket, és ezek esetében is lehetőségünk van a tetszőleges kombinálásra:
\begin{itemize}
	\item ExpectedImprovement
	\item ProbabilityOfImprovement
	\item LowerConfidenceBound
\end{itemize}

A GPflowOpt csomag fejlettségét mutatja, hogy lehetőséget ad az ismeretlen korlátok kezelésére is. Egy korlátfüggvényt definiálva beépíthető az algoritmusba, hogy a nyereség függvény azt is figyelembe vegye, az ismert korlátértékek alapján hol lehetnek a függvény által értelmezhető területek. Ezt tudja elvégezni a ProbabilityOfFeasibility osztály, amit az általunk használt egyéb nyereség függvénnyel kombinálhatunk. A működést azonban folytonos korlátfüggvényekre tervezték, így diszkrét értékkészletű függvények esetében, mint a mi "értelmezhető-nem-értelmezhető" bináris esetünk, nem ismert a függvény hatékonysága.

\subsubsection{MyGPflowOpt}
MyGPflowOpt osztályom célja a GPflowOpt által nyújtott testre szabható lehetőségeket kihasználva megoldani a kapott SPDN modell paraméter optimalizálását. A nyereség és kernel függvények közül az Acqisition és Kernel enum osztályok segítségével választhatunk, amelyek alapján aztán a MyAcquisiton és MyKernel segédosztályokat meghívva hozhatjuk létre a GPflow és GPflowOpt megfelelő objektumait.

% TODO !!!!  Meg kéne kezdeni még egyszer ezt a contrained alkalmazást :( Hátha hátha 

Mint említettem, ismeretlen korlátok kezelésére van lehetőség, azonban diszkrét esetre a működés nem kielégítő. Ezért ebben az implementációban is a büntető érték megoldását választottam a kiértékelhetetlen területek jelölésére és elkerülésének az ösztönzésére.

\subsection{Shogun toolbox könyvtárral - MyShogunOpt}

A Shogun Machine learning Toolbox széles tárházat biztosít a hatékony és egységes gépi tanuláshoz. Nyílt forráskódú, C-ben implementált, de számos programozási nyelven elérhető, így Pythonul is. Hatékony eszköz a Gauss folyamatok kezeléséhez, akár regresszióról, akár klasszifikációról van szó. Erre alapozva írtam meg saját Bayesi optimalizációmat. Lehetővé teszi a hiperparaméterek optimalizálását is, ezzel csökkentve annak a lehetőségét, hogy a rosszul megválasztott beállítások csökkentik az optimalizáció hatékonyságát.

\subsubsection{Shogun osztályok}

A Bayesi optimalizáció során szükségünk van számos Shogun osztályra a Gauss folyamatok kezeléséhez.
\paragraph{RealFeatures:}
Azon pontokat reprezentálja, melyek ismertek a függvénytérben, erre illesztjük a prior modellt. Inicializáláskor oszlopvektoros kétdimenziós tömbben várja a paraméterek értékeit.
\paragraph{RegressionLabels:}
Az ismert pontokhoz tartozó függvényértékeket kezelő osztály, regressziónál.
\paragraph{BinaryLabels:} 
Az ismert pontokhoz tartozó függvényértékeket kezelő osztály, bináris klasszifikációnál.
\paragraph{GaussianKernel:}
Gauss, vagy más néven négyzetes exponenciális kernel osztálya. Két paramétere van, az első a memória maximális mérete, nem jelentős, a második pedig a kernel működését befolyásoló paraméter ($\tau$).
\paragraph{ZeroMean:}
0 várhatóérték függvény.
\paragraph{GaussianLikelihood:}
A Gauss valószínűséget reprezentálja.
\paragraph{LogitLikelihood:}
Szigmoidhoz hasonló függvény, valószínűségi eloszlás, a bináris klasszifikációt segíti.
\paragraph{ExactiInferenceMethod:}
Létrehozza a Gauss folyamatot a megadott tulajdonságokkal, mátrix függvények kiszámításával. Paraméterei a kernel, az ismert pontok és értékeik, a várhatóérték függvény és egy valószínűségi eloszlás. 
\paragraph{SingleLaplaceInferenceMethod:}
A bináris klasszifikáció Gauss folyamatát állítja fel. Paraméterei a kernel,az ismert pontok és értékeik, a várhatóérték függvény és egy valószínűségi eloszlás.
\paragraph{GaussianProcessRegression:}
A regressziót végző osztály. Paramétere egy Gauss folyamat.
\paragraph{GaussianProcessClassification:}
A klasszifikációt végző osztály. Paramétere egy Gauss folyamat.
\paragraph{GradientCriterion:}
Hiperparaméterek optimalizálásához szükséges gradiens kritériumot reprezentáló osztály. Nem igényel paramétereket.
\paragraph{GradientEvaluation:}
Hiperparaméterek optimalizálásához szükséges osztály, paraméterei egy Gauss folyamat, az ismert pontok és értékeik, egy gradiens kritérium, és egy boolean érték az automatikus zárolásról (esetünkben False).
\paragraph{GradientModelSelection:}
Az optimális hiperparamétereket kiválasztó osztály, paramétere egy GradientEvaluation objektum.

\subsubsection{MyShogunOpt}

Az általam implementált Bayesi optimalizációt végző osztály a MyShogunOpt. Példányosításánál megadandó paraméterek:
\paragraph{model} Optimalizálandó SPDN modell
\paragraph{error\_value} Büntetőérték, ha nem kiszámítható helyen számítunk függvényértéket. Deafult értéke 10000.\\

A Gauss folyamatokból a regressziót az ismeretlen nemlineáris célfüggvényünk közelítéséhez, a klasszifikációt pedig a nem kiszámítható területek feltérképezéséhez használom. Mivel mind a klasszifikációt, mind a regressziót minden esetben ugyanazokra a pontokra számítjuk ki, egészen egyszerűen az SPDN objektumunk által adott eredményt vizsgáljuk: helyes érték esetén a klasszifikáció célfüggvényének a pontban való értéke 1, SPDNException esetében azonban -1. Ekkor a regressziós modell frissítéséhez az adott büntetőértéket használjuk.

A nyereség függvények közül a javulás várható értékét (\ref{eq:EI} képlet) használtam, megszorozva a klasszifikáció által visszaadott valószínűséggel, hogy az adott pontban a célfüggvényünk értelmezhető-e. A maximumának a megkereséséhez a SciPy könyvtár minimize metódusát használtam, a nyereség függvényem -1-szeresével.

Az algoritmus működése során az előző fejezetben leírt Shogun osztályokat alkalmaztam. Az algoritmus működése:

% TODO ide mindenképpen jó lenne egy folyamatábra
\begin{enumerate}
	\item Megadott számú kezdeti pontot mintavételezünk, az aktuális globális minimumot egy változóban tároljuk.
	\item Felállítjuk rá a Gauss folyamatokat, külön modellt a regresszióhoz, külön a klasszifikációhoz.
	\item Hiperparaméter optimalizálás elvégzése a meglévő információkkal.
	\item A felállított posterior modellek várhatóérték függvénye becslése a célfüggvényünknek.
	\item Nyereség függvény maximumának megkeresése.
	\item Az így kapott pontban a célfüggvény kiszámítása, globális minimum frissítése, ha annál jobbat találtunk. A pont és függvényérték hozzáadása az eddigi ismert pontokhoz és értékekhez. 
	\item Regressziós és klasszifikációs modell frissítése az új ponthalmazzal.
	\item Folytatjuk a 3. pontban, egészen addig, míg el nem végezzük a maximális iterációszámot, vagy a globális minimum bele nem esik a tűréshatárba, kellően közel 0-hoz.
\end{enumerate}











