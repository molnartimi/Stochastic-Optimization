\pagenumbering{roman}
\setcounter{page}{1}

\selecthungarian

%----------------------------------------------------------------------------
% Abstract in Hungarian
%----------------------------------------------------------------------------
\chapter*{Kivonat}\addcontentsline{toc}{chapter}{Kivonat}

Különböző modellekkel bármely tudományterületen találkozhatunk.  Mérnökökként a valós világot tetszőleges mértékben leegyszerűsítve végezzük feladatainkat. Sok esetben az események bekövetkezésének valószínűségét nem hanyagolhatjuk el, így sztochasztikus modellekkel kell dolgoznunk.

Sztochasztikus modell esetén valószínűségi eloszlásokkal, várható értékekkel, tüzelési valószínűségekkel számolunk. Parametrikus esetben ezek az átmeneti valószínűségek ismeretlen változók. Ilyen helyzettel állunk szemben bizonyos rendszerek tervezésénél, informatikusként például szolgáltatásbiztonság vagy teljesítmény tervezésénél, vizsgálatánál.

Markov láncról beszélünk, ha modellezett folyamataink jövője és múltja függetlenek a jelen ismeretében, ugyanakkor az állapotváltozások tetszőleges időpontban történhetnek.

A paraméterek valamilyen konkrét értékével a rendszer kimenete megfigyelhető, így, bár a rendszert leíró függvényt pontosan nem ismerjük, a függvényértékeket felhasználva végezhetünk műveleteket a kívánt cél elérése érdekében - ami a mi esetünkben az optimalizálás lesz. Paraméter szintézissel keressük azokat a paramétereket, melyekkel a modell kielégíti a specifikációban definiált elvárásokat.

A kulcslépés a valószínűségi modellek ellenőrzésénél az elérhetőségi valószínűség kiszámítása: mekkora eséllyel érünk el adott állapotokat? Nemdeterminizmus működést mutató modellek esetében, amilyen a Markov döntési folyamat is, ezen elérhetőségi valószínűségek az alapjai a nemdeterminizmus áthidalásának. Ennek segítségével iteratív módszereket alkalmazva lehetőségünk van a modell kiértékelésére.

Sok esetben azonban a modell kiértékelése nem az elsődleges célunk. Az élet minden területén az optimális megoldásokat keressük, így van ez egy rendszer tervezésénél is. 

Szakdolgozatomban arra a kérdésre keresem a választ, adott modell esetén mely paraméter lekötéssel érhető el az elvárt optimális viselkedés. Ennek a megválaszolásához számos optimalizáló algoritmust kipróbálok, saját és létező implementációkat vegyítve a ,,tökéletes'' módszer megtalálása érdekében.

A tanszéken fejlesztett PetriDotNet alkalmazásban leírt modelleket egy TDK keretein belül megalkotott megoldó segítségével értékelem ki, és különböző ismert optimalizáló algoritmusokkal keresem a modellek azon paramétereit, melyek a lehető legjobban közelítik a célfüggvényünk minimumát. Ez a célfüggvény a reward függvények elvárt értékektől való eltérésének négyzetössze. Célunk ezen összesített hibaérték lehető legkisebbre csökkentése.


\vfill
\selectenglish


%----------------------------------------------------------------------------
% Abstract in English
%----------------------------------------------------------------------------
\chapter*{Abstract}\addcontentsline{toc}{chapter}{Abstract}

% TODO


\vfill
\selectthesislanguage

\newcounter{romanPage}
\setcounter{romanPage}{\value{page}}
\stepcounter{romanPage}