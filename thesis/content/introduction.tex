%----------------------------------------------------------------------------
\chapter{\bevezetes}
%----------------------------------------------------------------------------

Az informatikában, akár a többi tudományterületen, elképzelhetetlen lenne a problémák vizsgálata, megoldása modellek, hipotézisek felállítása nélkül. Mérnökökként mi is a valós világot tetszőleges mértékben leegyszerűsítve végezzük feladatainkat. Sok esetben a világ egyik fontos tulajdonságát, az események bekövetkezésének valószínűségét nem hanyagolhatjuk el. A valószínűségszámítás alkalmazása során pedig gyakran kerülünk szembe matematikai időben lejátszódó véletlen jelenségekkel. Ezek leírásához használjuk a sztochasztikus folyamatok modelljét:

\begin{definition}
	Legyen $\mathcal{X}={X_t:t \in T}$ valószínűségi változóknak egy $t$ paraméterrel indexelt családja. $\mathcal{X}$-et \textbf{sztochasztikus folyamat}nak nevezzük. A $t$ paramétert (főleg az alkalmazásokban gyakori jelentése miatt) rendszerint az idővel azonosítjuk. Ha $T = {0,1,2,\dots}$, akkor $\mathcal{X}$-et diszkrét idejű folyamatnak vagy idősornak, ha pedig $T=[0,\infty)$, akkor folytonos idejű folyamatnak hívjuk. $X_t$-t diszkrét idejű folyamat esetén $X_n$-nel, folytonos idejű folyamat esetén pedig (a függvényszerű írásmóddal utalva a valós értékű paraméterre) $X(t)$-vel jelöljük. Azt az $S$ halmazt, melyből az $X(t)$ valószínűségi változók értékeiket veszik, \textbf{állapottér}nek (vagy állapothalmaznak) nevezzük.
\end{definition}

A sztochasztikus modellek számos problémakörben jelentenek hatékony megoldást, ahol az egyszerűbb esetekre alkalmazott determinisztikus megoldások nem vezetnek eredményre.
Ilyen helyzettel állunk szemben bizonyos rendszerek tervezésénél, informatikusként például szolgáltatásbiztonság vagy teljesítmény tervezésénél, vizsgálatánál.

Sok esetben rendszerünkre igaz a következő definíció is:

\begin{definition}
	Egy $\mathcal{X}$ sztochasztikus folyamatot \textbf{Markov-lánc}nak nevezünk, ha teljesül rá a Markov-tulajdonság, miszerint minden $n\ge1$ és $x_0,x_1,\dots x_{n-1},x_n\in S$ esetén
	$$P(X_n=x_n|X_{n-1}=x_{n-1},\dots X_0=x_0) = P(X_n=x_n|X_{n-1}=x_{n-1}),$$
	amennyiben a feltételes valószínűségek léteznek.\cite{MarkovLancokKonyv}
	Tehát szemléletesebben, modellezett folyamataink jövője és múltja függetlenek a jelen ismeretében. Ugyanakkor az állapotváltozások tetszőleges időpontban történhetnek.
\end{definition}

Sztochasztikus modell esetén valószínűségi eloszlásokkal, várható értékekkel, tüzelési valószínűségekkel számolunk. Parametrikus esetben ezek az átmeneti valószínűségek nem ismertek, ismeretlen változókként kezeljük őket. Amennyiben a  tüzelési valószínűségek exponenciális eloszlást követnek, az exponenciális eloszlás rátája függhet a modell paramétereitől, ezáltal befolyásolják a hatótényezők a rendszert.

A modellekhez reward függvényeket definiálva, melyek metrikákat, vagyis méréseket azonosítanak, tudjuk a rendszerrel szemben felállított követelményeket megfogalmazni -- például a valószínűsége egy rossz rendszerállapotba való érkezésnek legyen egy adott érték alatt. A lényege, hogy a paraméterek valamilyen konkrét értékével a rendszer kimenete megfigyelhető, így, bár a rendszert leíró függvényt pontosan nem ismerjük\footnote{black-box}, a függvényértékeket felhasználva végezhetünk műveleteket a kívánt cél elérése érdekében -- ami a mi esetünkben az optimalizálás lesz. Paraméter szintézisnek nevezzük a folyamatot, mellyel megkeressük azokat a paramétereket, melyekkel a modell kielégíti a specifikációban definiált elvárásokat.

A kulcslépés a valószínűségi modellek ellenőrzésénél az elérhetőségi valószínűség kiszámítása: mekkora eséllyel érünk el adott állapotokat? Nemdeterminizmus működést mutató modellek esetében, amilyen a Markov döntési folyamat is, ezen elérhetőségi valószínűségek az alapjai a nemdeterminizmus áthidalásának. A fejlett modell ellenőrzők támogatást nyújtanak ehhez, ma már léteznek sikeres alkalmazások százmillió állapotú modellek kiértékelésére is.

Szükséges tehát annak a racionális függvénynek a kiszámítása, mely kiszámítja az elérhetőségi valószínűséget egy parametrikus Markov-láncban.
Ehhez az ismeretlen átmenetet nem-determinisztikus választásokkal helyettesítjük. Az így kapott paramétermentes modell már könnyebben vizsgálható a megszokott analízis módszerekkel. Ez a megvalósítás teljesítményben felülmúlja a többi, parametrikus Markov-láncokat kiértékelő társait, és a szépsége, hogy nagyon változatos problémakörben alkalmazható, akár csak a mi esetünkben a Markov döntési folyamat paraméterszintézise során.\cite{ParameterSzintezisCikk}

A modellek kiértékeléséhez iteratív módszert alkalmazunk. Ezen módszerek az általános, nagy % lineáris ?
rendszerek megoldásához a tudományos számítástechnika számos területén egyre nagyobb népszerűségnek örvendenek. Habár ma már léteznek explicit megoldások, számos újabb hatékony iteratív megoldó % megoldó ?= solver
jelent meg, amiket a nagyon nagy lineáris rendszerek kiértékelésére való igény növekedésével egyre nagyobb érdeklődés vesz kerül.

Az iteratív megoldó egy kezdetleges, becsült megoldásból indulva konvergál a megoldáshoz, minden iterációban egyszerre egy vagy több komponenst módosítva, meghatározott sorrendben. Ezt az alap működést kombinálva más módszerekkel, meglehetősen sikeres eredményt érhetünk el.\cite{SolverKonyv}

Sok esetben azonban a modell kiértékelése nem az elsődleges célunk. Az élet minden területén az optimális megoldásokat keressük, így van ez egy rendszer tervezésénél is. Egy optimalizáló algoritmus, mely megmondja a tervezőnek, milyen tulajdonságokkal bír az a rendszer, mely megfelel az elvárásoknak, számos területen nagy segítség lehet.

Szakdolgozatomban erre keresem a lehető legjobb megoldást. Mivel ez napjainkban is jelentős probléma, nem meglepő, hogy számos létező implementáció létezik az egyes algoritmusokhoz a világhálón. Célom -- egyúttal az én erőforrásaimat is optimalizálva -- ezeket is felhasználva megtalálni a számunkra ,,tökéletes'' algoritmust.

A tanszéken fejlesztett \textbf{PetriDotNet} alkalmazásban leírt modelleket egy TDK keretein belül megalkotott megoldó keretrendszer, az SPDN\cite{SpdnTDK} segítségével értékelem ki, és a különböző ismert optimalizáló algoritmusokkal keresem a modellek azon paramétereit, ahol a célfüggvényünk értéke a lehető legjobban közelíti a minimumot. Ez a célfüggvény nemes egyszerűséggel a reward függvényeknek a követelményekben rögzített értékektől való eltérésének négyzetösszege:
\begin{equation}
	\label{eq:celfgv}
	f(p_1,p_2,...)=\sum_{i=1}^{rewards}\left( \hat{R}_i-R_i(p_1,p_2,...)\right) ^2.
\end{equation}
Szeretnénk eléri, hogy az ezzel a képlettel leírt összesített hibaérték minél jobban nullához konvergáljon.

Dolgozatom \ref{sec:optimalizacios-megkozelitesek}. fejezetében ismertetem az optimalizálás sztochasztikus rendszerek esetében fennálló jellegzetességeit és kihívásait, valamint felállítom a szempont rendszert, amelyeket figyelembe véve összehasonlíthatóak az algoritmusok.

\Aref{sec:algoritmusok}. fejezetben bemutatok néhány lehetséges megoldást, és elemzem, milyen tulajdonságokkal bírnak az algoritmusok, amik a mi esetünkben befolyással vannak a kívánt eredmény elérésére.

A dolgozat \ref{sec:implementaciok}. fejezetében olvashatunk a részletes megvalósításról, majd a \ref{sec:meresek}. fejezetben láthatjuk az implementált algoritmusok működését, értékelését és összehasonlítását \aref{sec:optimalizacios-megkozelitesek}. fejezetben definiált szempontrendszer alapján.